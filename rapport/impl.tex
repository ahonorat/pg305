\section{Choix d'implémentation} % (fold)
\label{sec:impl}

Le code a été segmenté pour permettre une lecture plus aisée de l'ordre dans lequel s'effectuent les diverses opérations. Le code du maître se trouve donc dans un fichier séparé de celui des esclaves. Le fonctionnement du thread de communication est regroupé dans une fonction séparée de celle des threads de travail. Des fonctions permettent de calculer le mot suivant dans l'ordre que nous avons fixé et de vérifier si un mot est le mot de passe. 

\subsection{Représentation du mot de passe}


\subsection{\'Eclatement des intervalles}

L'ordre utilisé pour organiser la liste des mots à tester est ici lexicographique par taille, c'est-à-dire que nous comparons d'abord tous les mots d'une certaine longueur, dans l'ordre lexicographique, pour ensuite pouvoir passer aux mots comportant une lettre de plus.

Un intervalle est composé de deux éléments: un mot de départ et le nombre de mots à vérifier. Ainsi, chaque thread, peut vérifier tout un intervalle, en utilisant la fonction calculant le mot suivant.

\subsection{Accès à la file de tâches}

La file des tâches est partagée entre les différents threads de travail. Aussi faut-il gérer la concurrence des threads qui veulent y accéder, d'où l'utilisation de sections critiques/atomiques (voir l'exemple \ref{critical}). Ainsi, chaque thread peut retirer une tâche de la file et l'exécuter sans crainte de dédoubler un travail. 

\begin{figure}
\begin{lstlisting}
	#pragma omp critical
	{
	  if( NOT list_empty( task_list ){
	    task_to_deal_with = pop( task_list );
	  }
	}
\end{lstlisting}
\caption{Section critique qui permet de récupérer une tâche}
\label{critical}
\end{figure}