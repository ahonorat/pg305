\section{Recherche de mot de passe} % (fold)
\label{sec:recherche_de_mot_de_passe}

\subsection{Directives de l'énoncé} % (fold)
\label{sub:enonce}

% subsection 'enonc_ (end)
Le programme devait effectuer une reracherche de mot de passe par force brute. L'utilisateur lance tout d'abord un programme appelé master (maître). Celui-ci lance $p-1$ processus appelés slave (esclave). Chaque processus esclave lance à son tout $t+1$ threads. Parmi ces threads, un servira aux communications avec le maître et les autres effectueront des tâches. Ces tâches sont en fait des intervalles de mots à vérifier. 



TODO: terminaison

\subsection{Automates de communication} % (fold)
\label{sub:automates_de_communication}

Les threads de communication du maître et des esclaves répondent aux automates des figures \ref{fig:master} et \ref{fig:slave}

\begin{figure}[h!]
\centering
% \includegraphics[width=0.8\textwidth]{master}
\caption{Automate du Master}
\label{fig:master}
\end{figure}


\begin{figure}[h!]
\centering
% \includegraphics[width=0.8\textwidth]{slave}
\caption{Automate du Slave}
\label{fig:slave}
\end{figure}
